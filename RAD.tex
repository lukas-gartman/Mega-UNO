%
%  Title       :  Requirements and Analysis Document
%  Authors     :  Alex Gerdes
%  Created     :  August 20, 2018
%
%  Purpose     :  Template RAD for TDA367/DIT212
%
%-------------------------------------------------------------------------------

\documentclass[12pt,a4paper]{scrartcl}

\usepackage[hyphens]{url}
\usepackage{hyperref}
\usepackage{xcolor}
\usepackage{graphicx}
\usepackage{mathpazo}

\title{Requirements and Analysis Document for \ldots}
\author{authors}
\date{date\\version}

\begin{document}

\maketitle

\section{Introduction}

Give some background and explain the purpose of this application. Describe 
the functionality of the application. Describe the stakeholders of the project,
highlight who will benefit from/use this particular application.

\subsection{Definitions, acronyms, and abbreviations}

Create a word list to avoid confusion and give a definition of every abbreviation
you use in the document.


\section{Requirements}

\subsection{User Stories}

Use the template from the course website and list all user stories here. It is
fine to have them in an spreadsheet (or other applications, such as Trello) at
first, but they must end up here as well.

These user stories should describe what the user will be able to do. Write 
the user stories in language of the customer, and give them a unique ID. List
the user stories in order of priority. 

You need to annotate an user story whether or not it is implemented. We need to
know which user stories are implemented, such that we can check this during the
oral presentation.

\subsection{Definition of Done}

In this section you list the acceptance criteria that are common for all user
stories. For example, the code should reviewed and tests, it should be under
version control, etc.

\subsection{User interface}

Include sketches, drawings and explanations of the application's user interface.
Describe the navigation between the different views. 


\section{Domain model}

Give a high level view overview of the application using a UML diagram.

\subsection{Class responsibilities}

Explanation of responsibilities of classes in diagram.


\section{References}

List all references to external tools, platforms, libraries, papers, etc. 


\end{document}
