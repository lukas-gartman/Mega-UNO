%
%  Title       :  System design document
%  Authors     :  Alex Gerdes
%  Created     :  August 20, 2018
%
%  Purpose     :  Template SDD for TDA367/DIT212
%
%-------------------------------------------------------------------------------

\documentclass[12pt,a4paper]{scrartcl}

\usepackage[hyphens]{url}
\usepackage{hyperref}
\usepackage{xcolor}
\usepackage{graphicx}
\usepackage{mathpazo}

\title{System design document for \ldots}
\author{authors}
\date{date\\version}

\begin{document}

\maketitle

\section{Introduction}

Give an introduction to the document and your application.

\subsection{Definitions, acronyms, and abbreviations}

Definitions etc. probably same as in RAD


\section{System architecture}

The most overall, top level description of your application. If your application
uses multiple components (such as servers, databases, etc.), describe their
responsibilities here and show how they are dependent on each other and how they
communicate (which protocols etc.)

You will to describe the `flow' of the application at a high level. What happens
if the application is started (and later stopped) and what the normal flow of
operation is. Relate this to the different components (if any) in your
application.


\section{System design}

Draw an UML package diagram for the top level for all components that you have
identified above (which can be just one if you develop a standalone application).
Describe the interfaces and dependencies between the packages. Describe how you
have implemented the MVC design pattern.

Create an UML class diagram for every package. One of the packages will contain
the model of your application. This will be the design model of your
application, describe in detail the relation between your domain model and your
design model. There should be a clear and logical relation between the two. Make
sure that these models stay in `sync' during the development of your application.

Describe which (if any) design patterns you have used.

The above describes the static design of your application. It may sometimes be
necessary to describe the \emph{dynamic} design of your application as well. You 
can use an UML \emph{sequence diagram} to show the different parts of your
application communicate an in what order.


\section{Persistent data management}

If your application makes use of persistent data (for example stores user
profiles etc.), then explain how you store data (and other resources such as 
icons, images, audio, etc.). 


\section{Quality}

\begin{itemize}
  \item Describe how you test your application and where to find these tests. If 
    applicable, give a link to your continuous integration.
  \item List all known issues.
  \item Run analytical tools on your software and show the results. Use for example:
    \begin{itemize}
      \item Dependencies: \href{http://stan4j.com/}{STAN} or similar.
      \item Quality tool reports, like \href{http://filehippo.com/download_pmd/}{PMD}.
    \end{itemize}
\end{itemize}

NOTE: Each Java, XML, etc. file should have a header comment: Author,
responsibility, used by ..., uses ..., etc.


\subsection{Access control and security}

If you applications has some kind of access control, for example a login, of
has different user roles (admin, standard, etc.), then explain how you 
application manages this. 


\section{References}

List all references to external tools, platforms, libraries, papers, etc. The
purpose is that the reader can find additional information quickly and use this
to understand how your application works.


\end{document}
